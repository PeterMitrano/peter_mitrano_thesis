\section{Conclusion}  \label{RSS:sec:conclusion}

This paper proposes a novel data augmentation method designed for trajectories of geometric state and action robot data. We introduce the idea that augmentations should be valid, relevant, and diverse, and use these to formalize data augmentation as an optimization problem. By leveraging optimization, our augmentations are not limited to simple operations like rotation and jitter. Instead, our method can find complex and precise transformations to the data that are valid, relevant, and diverse. Our results show that this method enables significantly better downstream task performance when training on small datasets. In simulated planar pushing, augmentation decreases the prediction error by 14\%. In simulated bimanual rope manipulation, the success rate with augmentation is 70\% compared to 47\% without augmentation. We also perform the bimanual rope manipulation task in the real world, which demonstrates the effectiveness of our algorithm despite noisy sensor data. In the real world experiment, the success rate improves from 27\% to 50\% with the addition of augmentation.

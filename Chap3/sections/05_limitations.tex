\section{Limitations} \label{RSS:sec:limitations}

First, our proposed method uses many hyper-parameters which may be difficult to tune. However, there are methods which can be used to automatically tune these parameters \cite{Augerino2020,AutoAugment}.

Additionally, there are problems and applications where the proposed objective functions do not ensure validity, relevance, and diversity. In these cases, the structure of our augmentation and projection procedures can remain, while new objective functions are developed. Another limitation is that our method is not compatible with image data. Much recent research in robotics has moved away from engineered state representations like poses with geometric information, and so there are many learning methods which operate directly on images. Although this is a limitation of the proposed method, many of the augmentations developed for images are also not applicable to problems in manipulation, even when images are used. For instance, pose detection, 3D reconstruction, semantic segmentation, and many other tasks may not be invariant to operations like cropping, flipping, or rotating. Creating an augmentation method for manipulation that is applicable to images is an open area for future research.

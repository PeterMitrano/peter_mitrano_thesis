\section{Introduction} \label{Intro:sec:Intro}

One of the most common assumptions in robotic manipulation is that the object being manipulated is rigid. That is, when the object moves, the entire object moves without deforming. The evolution of the state of the object over time, given the robot's actions, is called the dynamics, and planning for manipulation relies heavily on accurate dynamics models. Assuming rigidity simplifies the dynamics considerably, and so planning with rigid models is efficient and sometimes sufficient. However, many of the objects we want robots to manipulate are not actually rigid. For example: clothing, food, wires, hoses, thread, foams, composites like carbon fiber, paper products, fluids, granular materials, or even people. Even thick steel bars may not be accurately described as rigid if the task is to cut or grind them. Without rigidity, planning for manipulation becomes difficult because, at present, we do not have dynamics models that are accurate for a wide enough range of states, actions, and objects.

Instead, we have dynamics models that are reliable in some cases but not others. This is true for both physics-based analytic models and for learned models. For physics-based models, the model may have incorrect parameters for friction, damping, or stiffness. It may be completely lacking certain physical phenomena like torsional friction, air-resistance, or plastic deformation. It may also be intentionally simplified in order to make the computation faster, which can be essential for planning and control. Learning dynamics models from real world data can mitigate this issue  because the learned dynamics will at least be accurate in the situations in which we have collected training data. But when there is a lack of relevant training data or poor generalization, the learned dynamics may still be unreliable. Real world data collection is costly, can lack diversity, and doesn't scale easily. Thus, the problem of unreliable dynamics models is not yet solved simply by learning the model from real world data. Therefore, this dissertation is motivated by the research question ``How can robots use inaccurate models and limited real-world data to plan to manipulate deformable objects?''. By studying this question, I seek to expand the capabilities of robots to new kinds of manipulation and new kinds of objects.

This thesis addresses the problems of unreliable dynamic models when real-world data collection is expensive. While most prior work handles these challenges with fast replanning and/or exhaustive training in simulation, I propose methods for learning where a dynamics model is accurate and avoiding inaccurate regions in planning. I focus specifically on so-called deformable one-dimensional objects (DOOs), which includes objects like rope, hoses, cables, string, or potentially objects like cooked spaghetti or plant stems. These objects are, in some ways, the simplest type of deformable object, because their state can be compactly represented as a sequence of points, and they exhibit less self-occlusion than other deformable objects.

To make progress on the problem of learning and planning with unreliable dynamics, I leverage knowledge of physics, topology, and the fact that we do not need accurate dynamics models of the entire state-action space in order to accomplish a given task. To incorporate physics, I make extensive use of physics simulators as well as design heuristics and cost functions based on our knowledge of contact, gravity, and friction. To incorporate topology, I develop a mathematical categorization of multi-arm grasping of DOOs to make planning more efficient. To avoid modeling the entire state-action space, I focus on regions of the state-action space which are useful for the desired tasks.

This thesis makes the following contributions, organized into four chapters:

\begin{itemize}
    \item Learning a classifier to predict where a dynamics model is accurate, and what to do when the model is inaccurate.
    \item Data augmentation to enable online-learning of the classifier using only several hours of real world data.
    \item Online adaptation that focuses on easy-to-model dynamics, which enables fast sim2real adaptation.
    \item A topological signature and planning method for regrasping that expands DOO manipulation capabilities to new robot morphologies and new tasks.
\end{itemize}
